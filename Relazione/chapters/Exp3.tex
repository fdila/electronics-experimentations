\chapter*{Esperimento 3: Trasmissione con UART}
\addcontentsline{toc}{chapter}{Esperimento 3: Trasmissione con UART}

\section*{Obiettivo}
Trasmettere un array di numeri utilizzando la seriale e leggere questi numeri utilizzando MATLAB.

\section*{Svolgimento\footnote{Nella repository è il progetto Exp04}}
Tramite CubeMX andiamo ad impostare la periferica USART3 per trasmettere e ricevere dati, impostiamo il baudrate a 9600 e abilitiamo l'interrupt associato.

Nella fase di setup abilitiamo la seriale e abilitiamo anche l'interrupt associato alla ricezione di un byte.
\begin{minted}
[
frame=lines,
framesep=2mm,
baselinestretch=1.2,
fontsize=\footnotesize,
]{C}
//enable USART
USART3->CR1 |= USART_CR1_UE;
//enable RX interrupt
USART3->CR1 |= USART_CR1_RXNEIE;
\end{minted}

Quando riceveremo un byte sulla seriale entreremo nell'interrupt.
Controlliamo cosa ha sollevato l'interrupt andando a guardare il registro SR.
Se siamo entrati nell'interrupt perchè è stato ricevuto un byte andiamo a controllare cosa abbiamo ricevuto.
Se abbiamo ricevuto un 10 significa, per nostra convenzione, che MATLAB ha richiesto i dati. Andiamo quindi ad abilitare l'interrupt TXE, che viene sollevato ogni volta che il registro DR è vuoto.
Ogni volta che viene sollevato questo interrupt, se abbiamo ancora dati da trasmettere, andiamo a chiamare la funzione tx\_fun(), che restituisce il byte che dobbiamo inviare.
Mettiamo questo byte nel registro DR che fa quindi partire la trasmissione.

\begin{minted}
[
frame=lines,
framesep=2mm,
baselinestretch=1.2,
fontsize=\footnotesize,
]{C}
void USART3_IRQHandler(void)
{
  /* USER CODE BEGIN USART3_IRQn 0 */
	
	if (USART3->SR & USART_SR_TXE){
		if(tx_index < tx_length){
			USART3->DR = tx_fun();
		} else {
			tx_index = 0;
			USART3->CR1 &= ~USART_CR1_TXEIE;
		}
	}
	
	if (USART3->SR & USART_SR_RXNE){
		uint8_t comando;
		comando = USART3->DR;
		if(comando == 10) {
			USART3->CR1 |= USART_CR1_TXEIE;
		}
	}
	
	
  /* USER CODE END USART3_IRQn 0 */
  HAL_UART_IRQHandler(&huart3);
  /* USER CODE BEGIN USART3_IRQn 1 */

  /* USER CODE END USART3_IRQn 1 */
}
\end{minted}

Abbiamo 3 variabili globali per permettere la trasmissione: un buffer, l'indice a cui ci troviamo nel buffer e la lunghezza in byte del buffer.
Il buffer contiene interi senza segno a 16 bit, ma noi dobbiamo accedere ai singoli byte per poterli inviare sulla seriale.

Andiamo quindi a creare un puntatore a 8 bit all'indirizzo del buffer, e usiamo questo puntatore per accedere ai byte. Ogni volta che viene chiamata la funzione andiamo a prendere dalla memoria un byte e aumentiamo l'indice. L'indice ci dice a che punto siamo nel buffer inteso come numero di byte.

\begin{minted}
[
frame=lines,
framesep=2mm,
baselinestretch=1.2,
fontsize=\footnotesize,
]{C}
uint16_t tx_buffer[] = {10, 20, 30, 40, 50, 60, 70, 80, 90, 100};
uint8_t tx_index = 0;
uint8_t tx_length = sizeof(tx_buffer);

uint8_t tx_fun(void){
	
	uint8_t* pointer = (uint8_t*) tx_buffer;
	uint8_t to_tx = *(pointer + tx_index);
	
	if (tx_index < tx_length)
		tx_index++;
	
	return to_tx;
}
\end{minted}
