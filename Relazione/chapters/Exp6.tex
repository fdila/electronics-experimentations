\chapter{Lettura e plot di un'onda con trigger}
\label{chap:adc_trigger}

\section*{Obiettivo}
Leggere un onda generata tramite un generatore di funzioni, trasmettere i valori a MATLAB e plottarli usando il DMA sia per l'ADC che per la seriale. Aggiungiamo inoltre un trigger sull'ADC: acquisiremo i dati quando rileviamo un segnale più alto di un valore di trigger, tenendo in memoria anche un numero definito di valori di pretrigger.

\section*{Svolgimento\footnote{\href{https://github.com/fdila/electronics-experimentation/tree/main/Exp09}{Il codice è disponibile nella repository, cartella Exp09}}}

Per questo esperimento andremo ad utilizzare le periferiche ADC, TIM, USART e DMA impostate come in precedenza.
L'unica differenza è che useremo il DMA in modalità circolare: in questo modo possiamo acquisire in loop i dati e quando arriviamo alla fine del buffer il DMA torna in modo automatico all'inizio del buffer e inizia a sovrascrivere i dati vecchi.
In questo modo possiamo implementare un'acquisizione con l'ADC in modo da avere il numero di campioni desiderati incluso una parte del buffer di pretrigger.

Innanzitutto impostiamo le periferiche e iniziamo l'acquisizione dei dati.

\begin{minted}
[
frame=lines,
framesep=2mm,
baselinestretch=1.2,
fontsize=\footnotesize,
]{C}
/* Setup ADC */

	//Select 1 conversions for each sequence
	ADC1->SQR1 = 0;
	//Select channel 5 (PA5)
	ADC1->SQR3 = 0x0;
	ADC1->SQR3 |= ADC_SQR3_SQ1_0;
	ADC1->SQR3 |= ADC_SQR3_SQ1_2;
	//Set EOC flag at the end of each conversion
	ADC1->CR2 |= ADC_CR2_EOCS;
	//Disable ADC end of conversion interrupt
	ADC1->CR1 &= ~ADC_CR1_EOCIE;
	//Turn on scan mode
	ADC1->CR1 |= ADC_CR1_SCAN;
	//Turn on DMA mode!!
	ADC1->CR2 |= ADC_CR2_DMA;
	
	/* Setup UART */
	//Enable UART and RX interrupt
	USART3->CR1 |= USART_CR1_UE;
	USART3->CR1 &= ~USART_CR1_TCIE;
	USART3->CR1 |= USART_CR1_RXNEIE;
	//Turn on DMA on transmission
	USART3->CR3 |= USART_CR3_DMAT;
	
	/* setup DMA2 stream 0 - ADC */
	//set number of elements
	DMA2_Stream0->NDTR = SIZE;
	//set source peripheral address
	DMA2_Stream0->PAR = (uint32_t) &ADC1->DR;
	//set destination memory address
	DMA2_Stream0->M0AR = (uint32_t) buffer;
	//set half word memory data size
	DMA2_Stream0->CR |= DMA_SxCR_MSIZE_0;
	//set half word peripheral size
	DMA2_Stream0->CR |= DMA_SxCR_PSIZE_0;
	//enable transfer complete interrupt
	//DMA2_Stream0->CR |= DMA_SxCR_TCIE;
	//disable transfer complete interrupt
	DMA2_Stream0->CR &= ~DMA_SxCR_TCIE;
	//enable circ mode
	DMA2_Stream0->CR |= DMA_SxCR_CIRC;
	
	/* Setup DMA1 UART */
	//Disable DMA
	DMA1_Stream3->CR &= ~DMA_SxCR_EN;
	//set number of elements (multiply by 2 because we send bytes)
	DMA1_Stream3->NDTR = SIZE*2 + 2;
	//set source memory address
	DMA1_Stream3->M0AR = (uint32_t) buffer;
	//set destination peripheral address
	DMA1_Stream3->PAR = (uint32_t) &USART3->DR;
	//set byte memory data size
	DMA1_Stream3->CR &= ~DMA_SxCR_MSIZE_0;
	DMA1_Stream3->CR &= ~DMA_SxCR_MSIZE_1;
	//set byte peripheral size
	DMA1_Stream3->CR &= ~DMA_SxCR_PSIZE_0;
	DMA1_Stream3->CR &= ~DMA_SxCR_PSIZE_1;
	//enable transfer complete interrupt
	DMA1_Stream3->CR |= DMA_SxCR_TCIE;

	
	//Turn on ADC
	ADC1->CR2 |= ADC_CR2_ADON;
	//reset ADC SR
	ADC1->SR = 0x0;
	//Set number of elements
	DMA2_Stream0->NDTR = SIZE;
	//Enable DMA2 
	DMA2_Stream0->CR |= DMA_SxCR_EN;
	//Enable ADC DMA bit
	ADC1->CR2 |= ADC_CR2_DMA;
	ADC1->CR2 |= ADC_CR2_DDS;
	//Enable TTM2 (start ADC conversion)
	TIM2->CR1 |= TIM_CR1_CEN;
\end{minted}

Abbiamo un buffer di una lunghezza SIZE predefinita.
Abbiamo inoltre una lunghezza predefinita di PRETRIGGER.

Quando riceviamo il byte '10', ovvero una richiesta di dati da parte di MATLAB andiamo ad abilitare l'interrupt EOC dell'ADC.

Nella routine di questo interrupt andiamo a controllare se il valore appena convertito dall'ADC è superiore alla soglia di trigger da noi voluta.
In quel caso andiamo a memorizzare l'indice al quale ci troviamo nel buffer. 
Andiamo poi ad acquisire altri $SIZE - PRETRIGGER $ dati.
Distinguiamo due casi:
\begin{itemize}
    \item L'indice del trigger è maggiore del numero di PRETRIGGER
    \item L'indice del trigger è minore del numero di PRETRIGGER
\end{itemize}
A seconda del caso in cui ci troviamo andiamo a calcolare qual è il valore di NDTR che dobbiamo far raggiungere al DMA prima di interrompere l'acquisizione e inviare i dati.

\begin{minted}
[
frame=lines,
framesep=2mm,
baselinestretch=1.2,
fontsize=\footnotesize,
]{C}
void ADC_IRQHandler(void)
{
  /* USER CODE BEGIN ADC_IRQn 0 */
	trigger_val = ADC1->DR;
	if(trigger_val > 1200 && data_ok == 0){
		trigger_buffer_index = SIZE - DMA2_Stream0->NDTR + 1;			
		if (trigger_buffer_index > PRETRIGGER){

			stop_ndtr = SIZE - (trigger_buffer_index - PRETRIGGER);
			//indice da mandare a matlab per inizio dei dati da plottare
			buffer[SIZE] = trigger_buffer_index - PRETRIGGER;
		} else{
			stop_ndtr = SIZE - (SIZE - (PRETRIGGER-trigger_buffer_index));
			buffer[SIZE] =  SIZE - (PRETRIGGER-trigger_buffer_index);
		}
		data_ok = 1;
	} else if (data_ok == 1){
		if (DMA2_Stream0->NDTR == stop_ndtr){
			//stop ADC and DMA
			//disable adc interrupt
			ADC1->CR1 &= ~ADC_CR1_EOCIE;
			//disable tim2
			TIM2->CR1 &= ~TIM_CR1_CEN;
			//reset timer
			TIM2->CNT = 0x0;
			TIM2->SR = 0x0;
			//disable ADC DMA bit
			ADC1->CR2 &= ~ADC_CR2_DDS;
			ADC1->CR2 &= ~ADC_CR2_DMA;
			//disable DMA2
			DMA2_Stream0->CR &= ~DMA_SxCR_EN;
			//Reset DMA2 SR
			DMA2->LIFCR |= DMA_LIFCR_CTCIF0;
			DMA2->LIFCR |= DMA_LIFCR_CHTIF0;
			DMA2->LIFCR |= DMA_LIFCR_CTEIF0;
			//enable Transmission
			DMA1_Stream3->M0AR = (uint32_t) buffer;
			//Clear USART TC bit
			USART3->SR &= ~USART_SR_TC;
			//enable DMA1 (UART)
			DMA1_Stream3->CR |= DMA_SxCR_EN;
			stop_ndtr = 0;
			data_ok = 0;
			trigger_val = 0;
		}
	}
  /* USER CODE END ADC_IRQn 0 */
}
\end{minted}




Dopo aver finito l'acquisizione invieremo a MATLAB l'array dei dati convertiti e inoltre invieremo come ultimo elemento l'indice nel buffer al quale abbiamo rilevato il trigger.

In MATLAB andremo poi a fare il plot dei dati sfruttando l'indice inviato dal microcontrollore.

\begin{minted}
[
frame=lines,
framesep=2mm,
baselinestretch=1.2,
fontsize=\footnotesize,
]{MATLAB}
function plotUART(data)
    index = data(1001);
    new_data = [data(index:1000), data(1:(index-1))];
    plot(new_data);
end
\end{minted}